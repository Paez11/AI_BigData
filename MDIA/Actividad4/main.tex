\documentclass[12pt]{article}

\usepackage[utf8]{inputenc}
\usepackage[T1]{fontenc}
\usepackage[spanish]{babel}
\usepackage{graphicx}
\usepackage{listings}
\usepackage{caption}
\usepackage{subcaption}
\usepackage[right=2cm,left=2cm,top=2cm,bottom=2cm]{geometry}
\usepackage{hyperref}
\usepackage{fancyhdr}
\usepackage{color}
\usepackage[export]{adjustbox}
\usepackage{graphicx}
\usepackage{float}
\usepackage{changepage}
\usepackage{multicol}
\usepackage{imakeidx}
\usepackage[spanish]{babel}
\usepackage[backend=biber]{biblatex}

\pagestyle{fancy}
\renewcommand{\footrulewidth}{0.4pt}
\setlength{\headheight}{15pt}


\fancyhead[L]{ CEIABD – MIA }
\fancyhead[R]{ Páez Anguita, Víctor }
\fancyfoot[L]{IES Gran Capitán}


\begin{document}

\begin{titlepage}
    \begin{center}
      \Large \bfseries{}
    \end{center}
    \vspace{0.1cm}
    \begin{center}
      \Large \bfseries{}
    \end{center}
    \vspace{0.1cm}
    \begin{center}
     \Large \bfseries{Actividad 4}
    \end{center}
    \vspace{0.0001cm}
    \begin{center}
        Departamento de informática \\ I.E.S. Gran Capitán - Córdoba
    \end{center}
        \vspace{2 cm}
\begin{figure}[h!]
    \centering
    \includegraphics[width=.8\textwidth]{images.png}
    \label{fig:my_label}
\end{figure}
    \vspace{0.2 cm}
    \begin{center}
        Inteligencia artificial y Big data \\ Córdoba, 14 de Octubre 2024
    \end{center}
    \vspace{4 cm}
\null\hfill \textbf{Desarrollado por:}
\\
\\
\null\hfill Víctor Páez Anguita
\clearpage
\end{titlepage}

%%%%%%%%%%%%%%%%%%%%%%%%%%%Index%%%%%%%%%%%%%%%%%%%%%%%%%%%%%%%%
\tableofcontents
\clearpage
%%%%%%%%%%%%%%%%%%%%%%%%%%%Index%%%%%%%%%%%%%%%%%%%%%%%%%%%%%%%%

\section{Introducción}
En esta actividad realizaremos una entrevista a la inteligencia artificial mediante ChatGPT, y repetir las mismas preguntas al final del curso para ver cómo va evolucionando.

\section{Entrevista}
Aquí la entrevista que se nos a proporcionado del año pasado y la que he realizado:
\begin{itemize}
    \item \href{https://chatgpt.com/share/907d545d-40e9-4f90-8d3d-d2fbeffab0c5}{Entrevista del año pasado}.
    \item\href{https://chatgpt.com/share/670c71fa-8d74-8010-9bd9-caebf7366706}{Mi entrevista}.
\end{itemize}

\section{preguntas}
\subsection{¿Qué te ha parecido la entrevista?}

Bastante interesante, en comparación con la entrevista del año pasado se puede observar como el chatbot cada vez responde de una manera más natural en vez de algo más artificial. Algo que se acerca más 
a lo que es un robot. También se puede apreciar como las respuestas son más precisas y amplias. Da mucho que pensar como de rápido ha evoluciona el chat y de lo que será capaz de aquí a otro año.

\subsection{¿Qué tipo de futuro crees que nos espera con la revolución de la IA?}

Sigo manteniendo mi postura desde el momento en el que esta pregunta se mostro más presente desde que salieron las últimas actualizaciones de la IA como chatgpt o canvas. Para mi simplemnte es una nueva herramienta
más potente que la anterior que teniamos (internet). Pero sigue siendo eso, una herramienta. Creo que simplemnte simplificará muchas tareas y facilitará otras. Muy posiblemente muchos trabajos reducirán la 
demanda debido a la IA, pero en contraparte generará otros.

\clearpage

\section{Bibliografia}
\begin{itemize}
    \item \href{https://openai.com/index/chatgpt/}{Openai}.
\end{itemize}

\end{document}
