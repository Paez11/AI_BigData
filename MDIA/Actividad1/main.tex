\documentclass[12pt]{article}

\usepackage[utf8]{inputenc}
\usepackage[T1]{fontenc}
\usepackage[spanish]{babel}
\usepackage{graphicx}
\usepackage{listings}
\usepackage{caption}
\usepackage{subcaption}
\usepackage[right=2cm,left=2cm,top=2cm,bottom=2cm]{geometry}
\usepackage{hyperref}
\usepackage{fancyhdr}
\usepackage{color}
\usepackage[export]{adjustbox}
\usepackage{graphicx}
\usepackage{float}
\usepackage{changepage}
\usepackage{multicol}
\usepackage{imakeidx}
\usepackage[spanish]{babel}
\usepackage[backend=biber]{biblatex}

\pagestyle{fancy}
\renewcommand{\footrulewidth}{0.4pt}
\setlength{\headheight}{15pt}


\fancyhead[L]{ CEIABD – MIA }
\fancyhead[R]{ Páez Anguita, Víctor }
\fancyfoot[L]{IES Gran Capitán}


\begin{document}

\begin{titlepage}
    \begin{center}
      \Large \bfseries{}
    \end{center}
    \vspace{0.8cm}
    \begin{center}
      \Large \bfseries{}
    \end{center}
    \vspace{0.8cm}
    \begin{center}
     \Large \bfseries{Actividad 1}
    \end{center}
    \vspace{0.0001cm}
    \begin{center}
        Departamento de informática \\ I.E.S. Gran Capitán - Córdoba
    \end{center}
        \vspace{2 cm}
\begin{figure}[h!]
    \centering
    \includegraphics[width=.9\textwidth]{ia.jpg}
    \label{fig:my_label}
\end{figure}
    \vspace{0.2 cm}
    \begin{center}
        Inteligencia artificial y Big data \\ Córdoba, 10 de Octubre 2024
    \end{center}
    \vspace{6 cm}
\null\hfill \textbf{Desarrollado por:}
\\
\\
\null\hfill Víctor Páez Anguita
\clearpage
\end{titlepage}

%%%%%%%%%%%%%%%%%%%%%%%%%%%Index%%%%%%%%%%%%%%%%%%%%%%%%%%%%%%%%
\tableofcontents
\clearpage
%%%%%%%%%%%%%%%%%%%%%%%%%%%Index%%%%%%%%%%%%%%%%%%%%%%%%%%%%%%%%

\section{Introducción}
En esta actividad veremos unos cuantos casos de los malos usos que actualmente se le pueden dar a la inteligencia artificial y todavía no hay regulación para estos.

\section{Malos usos de la IA}

\subsection{Deepfake}

El uso indebido de la inteligencia artificial (IA) ha generado preocupaciones, como lo demuestra un ejemplo de 2017 en el que se creó un vídeo "deepfake" de Obama, sincronizando los labios del expresidente con la voz de un actor. Aunque antes este tipo de videos requería altos recursos técnicos, hoy en día están al alcance de cualquier persona con mínimos conocimientos y herramientas accesibles.

Los vídeos ultrafalsos representan un desafío para las democracias, ya que facilitan la creación de contenido engañoso. También son una amenaza a nivel individual, especialmente para las mujeres, ya que la mayoría de los deepfakes que circulan en internet son de pornografía no consentida. Existen incluso sitios que permiten "desnudar" a una persona en una foto por pocos dólares.

Además, se pueden usar avatares humanos generados por IA para actividades legítimas como la publicidad, pero también para fraudes o chantajes, como suplantar a personas famosas o crear identidades falsas en servicios digitales. De manera similar, modelos de lenguaje como GPT-4, según la agencia de verificación Newsguard, pueden ser utilizados para generar desinformación, discursos de odio, ciberacoso, o engaños sofisticados mediante chatbots.

\subsection{plagio y fraude de la IA generativa}

La inteligencia artificial generativa, como ChatGPT, es muy útil para automatizar tareas, crear contenidos, redactar informes o corregir textos. Sin embargo, su disponibilidad al público ha generado conflictos relacionados con los derechos de autor y la creación de contenido original.

Estas IA son ampliamente utilizadas por estudiantes para hacer trabajos y por empresas para generar contenido web, especialmente en marketing. También las IA generadoras de imágenes, como Midjourney o Dall-E, están sustituyendo a ilustradores y fotógrafos, lo que ha afectado a estos profesionales.

El problema principal es que estas IA se entrenan con contenidos protegidos por derechos de autor sin autorización ni compensación para los creadores originales. Aunque no se puede hablar directamente de plagio, el uso indebido de estas herramientas genera preocupación. Esto podría llevar a una pérdida de empleos en los sectores creativos y, a largo plazo, a una disminución de la originalidad y la calidad, ya que las IA se alimentarían de contenidos creados por otras IA, afectando así a la verdadera creatividad.

\subsection{Suplantación de la identidad}

En nuestras actividades diarias en internet, dejamos una huella digital que puede ser aprovechada por terceros para suplantar nuestra identidad. Esta información, conocida como OSINT (fuentes abiertas), incluye datos que proporcionamos activamente, como fotos, reseñas y comentarios, o que quedan expuestos en brechas de seguridad.

La inteligencia artificial ha avanzado hasta permitir la creación de simulaciones de voz y video, conocidas como deepfakes. Los ciberdelincuentes pueden usar estos datos para generar voces y videos que imiten a una persona real. Un ejemplo es el caso de un ejecutivo británico que fue engañado por un estafador que usó una voz generada por IA, imitando a su jefe, para convencerlo de transferir 243.000 dólares.

Otro riesgo es el uso de videos generados para superar controles de seguridad, como videollamadas de verificación bancaria, donde la IA imitaría expresiones faciales y gestos de la persona suplantada, desafiando sistemas de identificación como los controles KYC (Know Your Customer).

Este resumen subraya los riesgos de privacidad y seguridad vinculados al uso de la IA para suplantación de identidad y cibercrimen.

\section{Otros usos}
\subsection{Aportación de información peligrosa}

Hemos visto como chat gpt puede ser muy útil a la hora de obtener información sobre ciertos campos tecnología, ciencia, cocina, etc. Igual de potente como puede ser una busqueda por internet. Sin embargo, a diferencia de motores de busqueda como google o mozilla los cuales tienen protegidas ciertas busquedas para proteger y evitar el mal uso del conocimiento, los chatbots todavía tienen ciertas barreras que poner ante esto. Si bien es cierto que no proporcionan este tipo de información indebida para el usuario sigue siendo muy fácil sortear este tipo de seguridad. Simplemente insistiendo al chatbot y escribiendo cosas como que es para un proyecto o simplemente para información, finalmente el chatbot acabará escribiendo lo que quiere al usuario. Desde como fabricar cloroformo, realizar fraude fiscal e incluso casos de como fabricar un reactor nuclear.
\\
Aunque parezca que el conocimiento debe ser libre y para todo el mundo esto sigue siendo un tema contraversial y depende del uso que le de el usuario.

\clearpage
\section{Bibliografia}
\begin{itemize}
    \item \href{https://www.iic.uam.es/noticias/malos-usos-de-inteligencia-artificial-como-hacerles-frente/}{Articulo del instituto de tecnología del conocimiento}.
    \item \href{https://protecciondatos-lopd.com/empresas/mal-uso-inteligencia-artificial/}{Articulo del Grupo Ático 34}
    \item \href{https://www.navascusi.com/riesgos-inteligencia-artificial-usurpacion-identidad/}{Articulo de Navas \& cusi}
\end{itemize}

\end{document}
