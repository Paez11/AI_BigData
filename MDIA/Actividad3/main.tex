\documentclass[12pt]{article}

\usepackage[utf8]{inputenc}
\usepackage[T1]{fontenc}
\usepackage[spanish]{babel}
\usepackage{graphicx}
\usepackage{listings}
\usepackage{caption}
\usepackage{subcaption}
\usepackage[right=2cm,left=2cm,top=2cm,bottom=2cm]{geometry}
\usepackage{hyperref}
\usepackage{fancyhdr}
\usepackage{color}
\usepackage[export]{adjustbox}
\usepackage{graphicx}
\usepackage{float}
\usepackage{changepage}
\usepackage{multicol}
\usepackage{imakeidx}
\usepackage[spanish]{babel}
\usepackage[backend=biber]{biblatex}

\pagestyle{fancy}
\renewcommand{\footrulewidth}{0.4pt}
\setlength{\headheight}{15pt}


\fancyhead[L]{ CEIABD – MIA }
\fancyhead[R]{ Páez Anguita, Víctor }
\fancyfoot[L]{IES Gran Capitán}


\begin{document}

\begin{titlepage}
    \begin{center}
      \Large \bfseries{}
    \end{center}
    \vspace{0.1cm}
    \begin{center}
      \Large \bfseries{}
    \end{center}
    \vspace{0.1cm}
    \begin{center}
     \Large \bfseries{Actividad 3}
    \end{center}
    \vspace{0.0001cm}
    \begin{center}
        Departamento de informática \\ I.E.S. Gran Capitán - Córdoba
    \end{center}
        \vspace{2 cm}
\begin{figure}[h!]
    \centering
    \includegraphics[width=.8\textwidth]{portada.png}
    \label{fig:my_label}
\end{figure}
    \vspace{0.2 cm}
    \begin{center}
        Inteligencia artificial y Big data \\ Córdoba, 14 de Octubre 2024
    \end{center}
    \vspace{4 cm}
\null\hfill \textbf{Desarrollado por:}
\\
\\
\null\hfill Víctor Páez Anguita
\clearpage
\end{titlepage}

%%%%%%%%%%%%%%%%%%%%%%%%%%%Index%%%%%%%%%%%%%%%%%%%%%%%%%%%%%%%%
\tableofcontents
\clearpage
%%%%%%%%%%%%%%%%%%%%%%%%%%%Index%%%%%%%%%%%%%%%%%%%%%%%%%%%%%%%%

\section{Introducción}
En esta actividad seguiremos el ejemplo de la actividad anterior, para esta vez, ver ejemplos en el ambito de reconocimiento de voz y el procesamiento del lenguaje natural.

\section{Aplicación de reconocimiento de voz y procesamiento del lenguaje natural}

\subsection{Industria del cine}

Asistentes inteligentes para subtitulado y doblaje: El procesamiento del lenguaje natural (PLN) y el reconocimiento de voz están revolucionando el subtitulado automático y el doblaje en películas y series. 
Mediante el uso de asistentes que pueden transcribir el habla, ajustar diálogos y traducir automáticamente en múltiples idiomas, se facilita el trabajo de producción y localización. Esto no solo ahorra tiempo, 
sino que mejora la accesibilidad para diferentes públicos.
\\
Creación de guiones: En la etapa de preproducción, herramientas basadas en PLN pueden ayudar a los guionistas a generar ideas y escribir diálogos más naturales gracias a la comprensión del lenguaje humano.

\subsection{Aplicaciones industriales}

Automatización de atención al cliente: En sectores como la banca o el comercio electrónico, el reconocimiento de voz junto con el PLN permite que los sistemas de atención
al cliente automatizados entiendan consultas complejas y ofrezcan respuestas personalizadas. Esto reduce la necesidad de intervención humana y mejora la eficiencia del servicio.
\\
Control de maquinaria y procesos: Los comandos de voz están integrándose en la gestión de máquinas industriales y procesos de fabricación, permitiendo a los trabajadores interactuar con
la maquinaria de forma más natural y eficiente, reduciendo tiempos de inactividad.

\subsection{Vida Cotidiana y Social}

Asistentes virtuales: Herramientas como Siri, Alexa o Google Assistant utilizan reconocimiento de voz y PLN para permitir a los usuarios controlar sus dispositivos, 
realizar búsquedas o gestionar tareas diarias mediante comandos hablados. Estas interacciones están cada vez más personalizadas a los patrones de voz y comportamientos de los usuarios.
\\
Domótica: El reconocimiento de voz aplicado al control de dispositivos del hogar es un gran avance en accesibilidad, especialmente para personas con movilidad reducida. 
Desde encender luces hasta ajustar la temperatura, estos sistemas hacen las viviendas más inteligentes y accesibles.

\clearpage
\section{Otros usos}
\subsection{Aplicaciones en medicina}
Documentación médica: Los profesionales de la salud utilizan reconocimiento de voz para dictar notas clínicas durante el tratamiento de pacientes. 
Esto permite a los médicos registrar información crucial sin interrumpir el proceso de atención.
\\
Diagnóstico por voz: Investigaciones recientes están explorando el uso del reconocimiento de voz para detectar enfermedades a partir de patrones vocales, 
como posibles indicadores de trastornos neurodegenerativos o respiratorios.

\subsection{Educación y aprendizaje}

Herramientas de transcripción en tiempo real: En aulas y conferencias, se utilizan sistemas de reconocimiento de voz para transcribir clases en tiempo real, 
lo que es útil para estudiantes con discapacidades auditivas o para quienes prefieren revisar el contenido más tarde.
\\
Asistentes educativos: Programas como Duolingo utilizan el procesamiento del lenguaje para entender la pronunciación y mejorar las habilidades lingüísticas de los estudiante.

\clearpage

\section{Bibliografia}
\begin{itemize}
    \item \href{https://orbitainternet.com/el-reconocimiento-de-voz-con-inteligencia-artificial-como-funciona-y-su-impacto-en-nuestra-vida-diaria/}{Articulo de orbitainternet}.
    \item \href{https://www.questionpro.com/blog/es/procesamiento-del-lenguaje-natural/}{Articulo de questionpro}
    \item \href{https://blog.pareto.io/es/reconhecimento-por-voz/}{Articulo de pareto.io}
\end{itemize}

\end{document}
